\begin{block}{Observed Circulation Anomalies}
  \begin{mdframed}
  \begin{figure}
    \caption{
  		Daily rainfall averaged over Lower Paraguay River Basin and observed weather type for each day in NDJF 2015-16.
      Blue lines indicate the climatological 50th, \nth{90}, and 99th percentiles of NDJF area-averaged rain.
      \label{fig:rain-wt}
  	}
    \noindent\includegraphics[width=0.925\textwidth]{wt-rain-time-series.pdf}
    \end{figure}
  \end{mdframed}

  During austral summer (NDJF) 2015-16, most heavy rainfall occurred during weather types 1 and 4 (\cref{fig:rain-wt}).
  Monthly-scale circulation anomalies (\cref{fig:anomalies}) show a weak anticyclonic circulation that set up over central Brazil during November 2015 and strengthened into the following month.
  In January 2016 it weakened before returning in February 2016.
  The observed rainfall and circulation anomalies are consistent with the aggregation of the observed weather types shown in \cref{fig:rain-wt}.

  \begin{mdframed}
  \begin{figure}
  	\noindent\includegraphics[width=0.925\textwidth]{circulation-NDJF-1516-anomaly-alt.pdf}
  	\caption{
      	Monthly composite anomalies observed during NDJF 2015-16.
        Colors: anomalies of rainfall associated with each weather type [\si{\milli\meter\per\day}].
        Contours: anomalies of \SI{850}{\hecto\pascal} streamfunction [contour interval \SI{1e4}{\meter\squared\per\second}]
        \label{fig:anomalies}
  	}
  \end{figure}
  \end{mdframed}
\end{block}
