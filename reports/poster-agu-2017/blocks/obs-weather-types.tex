\begin{block}{Observations and Weather Types}
  Observations come from:
  \begin{itemize}
    \item Rainfall: CPC Global Unified \cite{xie2010cpc}
    \item Atmosphere: NCAR-NCEP Reanalysis II \cite{Kanamitsu:2002kk}
  \end{itemize}
  We use weather typing \cite{Munoz2015} to represent daily circulation patterns:
  \begin{enumerate}
    \item Calculate streamfunction $\Psi$ from meridional and zonal wind \cite{Dawson:2016ge}
    \item Project \SI{850}{\hecto\pascal} streamfunction onto leading 4 EOFs
    \item $K$-means clustering using classifiability index \cite{Michelangeli1995} to generate single weather type for each day
  \end{enumerate}
  Weather typing simplifies dynamics of daily rainfall but \emph{facilitates analysis of sequences of daily weather patterns}.
  They are associated with patterns that have been well described in the literature; particularly relevant are:
  \begin{itemize}
    \item WT1 represents ``Chaco'' jet event \cite{Salio:2002ev}
    \item WT4 represnts ``No-Chaco'' jet events \cite{Vera:2006ib}
  \end{itemize}
  \begin{mdframed}
  \begin{figure}
  	\noindent\includegraphics[width=0.925\textwidth]{weather-type-composite-alt.pdf}
  	\caption{
  		Colors: anomalies of rainfall associated with each weather type [\si{\milli\meter\per\day}].
      Contours: anomalies of \SI{850}{\hecto\pascal} [contour interval \SI{1e4}{\meter\squared\per\second}]
  	}
    \label{fig:weather-types}
  \end{figure}
  \end{mdframed}
% \item S2S forecasts issued by ECMWF \cite{Vitart2016}
\end{block}
