\begin{block}{Circulation Patterns Associated with RIP Events}
  GCMs represent observed dipole pattern, with some latitudinal bias in storm track and geometry of high pressure system
  \begin{figure}[ht]
    \centerline{\includegraphics[width=0.95\textwidth]{figs/CPC_RIPs_MAM_ZG_700_PR_WAT.pdf}}
    \caption{
      Daily composites of $Z_{700}$ anomalies (shades) and integrated precipitable water content anomaly (contours at \SI{3}{\kilo\gram\per\square\meter}) from four days before each spring (MAM) RIP event to one day following the event.
      Solid contours represent positive anomalies and dashed contours represent negative anomalies.
      An ``X'' indicates that at least 80\% of composite members had anomalies of the same sign in that location.
    }
    \label{fig:MAM-gph-pr-wat}
  \end{figure}
  \begin{figure}[ht]
    \centerline{\includegraphics[width=0.95\textwidth]{figs/GCM_OBS_RIPs_MAM_Z_700.pdf}}
    \caption{
      MAM RIP day composites of absolute $Z_{700}$ (shading) and $Z_{700}$ anomalies (contours in \SI{15}{\meter} increments) for 5 CM3 members and reanalysis (Obs).
    }
    \label{fig:MAM-mod-obs-gph}
  \end{figure}
\end{block}
