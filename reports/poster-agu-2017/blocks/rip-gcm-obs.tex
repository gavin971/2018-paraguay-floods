\begin{block}{RIP Events in a GCM \& Observations}
  \begin{itemize}
    \item CM3: too many MAM RIP days, too few back-to-back RIP days
    \item Seasonality bias: too many (few) RIP days in MAM (JJA)
    \item When CM3 produces intense precipitation in any part of the study region, it has a tendency to simultaneously produce intense precipitation in several grid cells
    \item Discrepancy between the GCM runs and the observed RIP records is even more stark when the observed precipitation data is used to calculate the 99th percentile thresholds for the model and RIP records
  \end{itemize}
  Conclusion: \textbf{relevant precipitation events not well-simulated by CM3} -- no to \textbf{Q1}.
  \begin{figure}[ht]
    \centering
    \includegraphics[width=0.95\textwidth]{RIP_CPC_mod_MAM_year.pdf}
    \caption{
      (a) 10-year moving average of the number of MAM RIP days by year for the observed record (black solid line), the five GFDL CM3 ensemble members (light dashed lines), and the ensemble mean (heavy dashed line).
      (b) The counts for the the number of MAM RIP days by year for the observed record (black solid line), the five GFDL CM3 ensemble members (light dashed lines), and the ensemble mean (heavy dashed line).
    }
    \label{fig:RIP-historic}
  \end{figure}
\end{block}
