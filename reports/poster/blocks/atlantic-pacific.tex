\begin{block}{Atlantic-Pacific Interaction and WT4}
    While the occurrence of \gls{wt}1 during NDJF 2015-16 is well explained by ENSO and MJO variability, these features alone do not explain the occurrence of \gls{wt}4, the ``No-Chaco'' jet event.
    Previous studies emphasize the importance of Pacific-Atlantic interaction for forecasting climate effects in this region~\cite{Barreiro:2017ct}.
    A persistent SST dipole in the central southern Atlantic Ocean favors the occurrence of \gls{wt}4 by blocking transient extratropical wave activity coming from the Pacific, facilitating transitions from Chaco jet events (WT 1) to No-Chaco jet events (WT 4) via enhanced low-level wind circulation from southern Brazil towards the Atlantic, and back to north-east Brazil and the Amazon (see \cref{fig:atlantic-pacific}) due to land-sea temperature contrasts.
    Composite analysis (not shown) of months with many \gls{wt}4 occurrences is consistent with the schematic shown here.
    \begin{framed}
        \begin{figure}
            \centering
            \includegraphics[width=\textwidth]{ChacoNoChacojet.pdf}
            \caption{
                Schematics of low-level jet events (red arrows) during austral summer of El Ni\~{n}o years.
            }\label{fig:atlantic-pacific}
        \end{figure}
    \end{framed}
\end{block}
