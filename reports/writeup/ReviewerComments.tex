\documentclass{article}


%Packages
\usepackage{
	siunitx, % units
  graphicx, % for images
	natbib,
	booktabs,
	enumitem
}
\usepackage[letterpaper]{geometry}
\renewcommand{\thefigure}{S\arabic{figure}}
\renewcommand{\thetable}{S\arabic{table}}
\usepackage[breaklinks=true,hidelinks]{hyperref}

\usepackage{mdframed}
\newenvironment{answer}{\begin{mdframed}}{\end{mdframed}}

%-------------------------------------------------------------------------------
% Title and Authors
%-------------------------------------------------------------------------------

\title{Response to Reviewers}
\author{James Doss-Gollin\and \'{A}ngel G. Mu\~{n}oz  \and Simon J. Mason \and Max Past\'{e}n }
\date{\today}

\begin{document}

\maketitle

We would like to thank the editors and reviewers for their specific and helpful comments.
In particular, reviewer suggestions as to the presentation of results and discussion of literature have helped us clarify our thinking and improve the clarity of our writing.

Per request by two reviewers we have added additional fields (\texttt{WHICH}) to some figures which necesiates minor changes to the discussion of those figures and the data sub-section.
This also provided us with the opportunity to revise all figure captions; all three reviewers noted that several figure captions contained errors.

Reviewer 3 gave three major comments which led to some additional analysis.
\texttt{Describe more later}.

Please find our responses to the specific comments below.

\section{Response to Reviewer 1 Comments}

\begin{enumerate}
	\item Lines 317-321: This should be shown by other atmospheric fields. For instance, wind field at 850hPa and vertically integrated moisture flux.
	\begin{answer}
		The reviewer is suggesting that we incorporate other variables into our plots of figure 3 in order to better show the relationship to the established literature.
		Do we want to do it?
	\end{answer}
	\item Lines 333-334: It would be convenient to include these percentages as subtitles in each panel of Fig. 4.
	\begin{answer}
		The reviewer is refering to the percentage of variance explained.
		This is easy to do and we will do it.
	\end{answer}
	\item Lines 334-337: This should be further proved. Fig. 4 does not provide enough evidence to support these statements.
	\begin{answer}
		The reviewer is refering to the large-scale anomalies associated with each EOF.
		I think that we can change the language here -- we don't want or need to make any strong statements about what these EOFs mean, and were just trying to provide some physical intuition about them.
	\end{answer}
	\item Line 361: Table S1 is not included in the supplemental material.
	\begin{answer}
		This is an erroneous reference, address it.
	\end{answer}
	\item Lines 555-559: This is not shown or discussed in the present manuscript.
	\begin{answer}
		The reviewer is refering to the statement that antecedent conditions plus persistent, heavy rainfall led to heavy flooding.
		It is correct that we did not show or discuss this; we can say that the persistent and intense rainfall led to flooding and may have been influenced by the antecedent conditions -- can add hydrological references if needed but not really necessary.
	\end{answer}
	\item	Lines 671-673: Check the authors of this reference (Kim et al. 2014).
	\begin{answer}
		We thank the reviewer for identifying the duplication of the authors
	\end{answer}
	\item Lines 853-856 (Fig. 3): One panel is missing. Streamfunction at 200hPa?
	\begin{answer}
		The streamfunction is shown at \SI{850}{\hecto\pascal} and the misleading caption has been corrected.
	\end{answer}
	\item	Lines 860-863 (Fig. 6): Yellow labels are not easy to read.
	\begin{answer}
		Let's look for a better color map -- none will be ideal here.
	\end{answer}
	\item	Lines 864-866 (Fig. 7): Streamfunction at 200hPa is not shown.
	\begin{answer}
		As for Fig. 3, the caption is incorrect and has been corrected.
	\end{answer}
\end{enumerate}

\section{Response to Reviewer 2 Comments}


\begin{enumerate}
	\item Weather type some times appears as "WT" and other as " weather type".  It is suggest to define it (I have not found) and to use WT along the text.
	\begin{answer}
		Suggestion taken, will implement.
	\end{answer}
	\item L61 - "Grimm and Tedeschi 2009"  is not a appropriated to explain the warm season systems in the region.
	\begin{answer}
		Let's change the reference
	\end{answer}
	\item L80 and L470 - Please, to include a reference on the influence of MJO over the convection in the region.
	\begin{answer}
		Let's add a good one
	\end{answer}
	\item L84 - The correct citation here is Marengo et al. 2012. Please, check.
	\begin{answer}
		The reviewer is correct, that is the paper we intended to cite.
	\end{answer}
	\item L163 - Why "retaining the seasonal cycle"?
	\begin{answer}
		We can be more clear here: let's say that we retain intra-seasonal variability or similar.
	\end{answer}
	\item L166-171 - The phrase "All codes used … packages" may be moved to the supplementary material.
	\begin{answer}
		This is certainly a long several sentences that doesn't collectively add much to the analysis.
		At the same time, it's important to me to support reproducible research  and open science by making codes available and citing the authors of important open source packages.
		Let's keep it but make it much shorter.
	\end{answer}
	\item L174-176 - "This region corresponds to the region over which given topography and previous studies 
(Barros et al. 2004; Bravo et al. 2011), one might 
" should be " In this region, given topography and previous studies 
(Barros et al. 2004; Bravo et al. 2011), one might..."
	\begin{answer}
		The reviewer has suggested a more clear wording.
	\end{answer}
	\item L195 -What is the meaning of the affirmation "physical mechanisms are more interpretable"?
	\begin{answer}
		I'll ask \'{A}ngel to answer, he may have strong feelings here
	\end{answer}
	\item L261 - change "empirical orthogonal functions (EOFs) ..."  to "EOFs ..." since it was defined in page 11.
	\begin{answer}
		I'm not seeing it on line 261, assuming the reviewer meant line 244 in which case agreed
	\end{answer}
	\item L855, L864, - "Top and middle rows …" should be "Top and bottom rows …"
	\begin{answer}
		Agreed, Reviewer 1 also noticed some errors in the figure captions which have been addressed.
	\end{answer}
	\item L342 - "past …" should be "southward …"
	\begin{answer}
		Yes, this is a more clear wording
	\end{answer}
	\item L344 - "before …" should be "northward …"
	\begin{answer}
		Agreed
	\end{answer}
	\item L345 - Looking Figure 5, WT3 resembles that one anomaly pattern generally associated with establishment or not of SACZ and called "seesaw" by some authors.  Please, see and cite, for example, Nogués-Peagle and Mo  (1997) and Liebmann et al. (2004).
	\begin{answer}
		Let's re-read these papers but at a glance this looks good.
		The question is how much to incorporate into discussion.
	\end{answer}
	\item L372 - It is not clear the phrase "During early January 2016 weather type 3 led to persistent low rainfall" since I noted in Fig. 6 persistent WT3 from middle to end January. Please, clarify.
	\begin{answer}
		Yes, this occurred from mid to late January, not early January 2016.
	\end{answer}
	\item L375 - The legend of Figure 7 says "200 hPa and 850 hPa". But, it is showing stream-function anomalies in only one atmospheric level. What is the level? Please, to correct the legend.
	\begin{answer}
		The reviewer is correct and the erroneous figure captions have been fixed.
	\end{answer}
	\item L412 and Figure 10 legend - The denominations of the sub-seasonal forecasts have different names. It is suggest to use "S2S model forecasts" for both text and Figure.
	\begin{answer}
		I believe that the reviewer is referring to the ``Raw'' model only?
	\end{answer}
	\item L418 - There is no "10b" in Figure 10. It is suggested to include the letters "(a), (b), (c), … (m), …" in Figure 10, which is clearer than that used, for example, in L423 "10, top row, first two columns …"
	\begin{answer}
		OK, I'm not a huge fan of this but here it may be helpful.
	\end{answer}
	\item L439 - change "Principal Component Regression …" to "PCRs" already defined previously.
	\begin{answer}
		Agreed
	\end{answer}
	\item L442 - change "Canonical Correlation Analysis" to "CCA".
	\begin{answer}
		Agreed
	\end{answer}
	\item L448 - "resolution and uncertainty (Ignorance Score; second row, fourth and fifth columns) and discrimination (2AFC score; middle row, fourth and fifth columns)" should be "resolution and uncertainty according Ignorance Score and discrimination according 2AFC (fig. 10)"
	\begin{answer}
		Yes, this is a more readable wording
	\end{answer}
	\item L481 -Check the reference to Marengo et al. (2004) in this part of text since there is not rainfall analysis in this paper.
	\begin{answer}
		This is true, though the Marengo paper shows OLR and similar.
		I think it's reaonable to argue that we have identified simlar mechanisms to what were identified in the literature but we should be more careful with wording as it's not only heavy rainfall but really heavy rainfall, thunderstorms, and mesoscale convective activity.
		For reference, this citation occurs at line 461 not 481.
	\end{answer}
	\item L464-466 - Consider to re-write the phrase.
	\begin{answer}
		The reviewer is refering to the sentence ``Understanding the mechanisms and possible...research''.
		Will re-write.
	\end{answer}
	\item L488 - and others - change "South American Low-Level Jet…" to "SALLJ" already previously defined.
	\begin{answer}
		Agree
	\end{answer}
	\item L503 - and others - change "anti-cyclonic" to "anticyclonic"
	\begin{answer}
		The reviewer's suggestion is more consistent with AMS guidelines \url{http://glossary.ametsoc.org/wiki/Anticyclonic_circulation} so we have changed it.
	\end{answer}
	\item L512 - change "South Atlantic" to "SACZ"
	\begin{answer}
		Agree
	\end{answer}
\end{enumerate}

\section{Response to Reviewer 3 Comments}

\subsection{Major Comments}

\begin{enumerate}
	\item The following claim in the paper is not well supported by the evidence provided (or maybe the evidence is not clearly reported).
	Line 29, Line 501 to 503 and Lines 564 to 565 - The presence of a dipolar SST anomaly in southern Atlantic during NDJF 2015-16 which enhances the frequency of occurrence of No-Chaco events during this flood period is not shown in any figure. These No-Chaco jet events incidentally are the most important weather types (along with Chaco jets) in causing the flood (Figure 6) so a figure supporting the existence of the dipolar SST anomaly is required.
	Not much further analysis is required, only a figure showing the SST anomalies in December 2015-16 will suffice.
	Also no evidence has been shown of a strong El Nino in this period. Just a monthly mean time series of Nino 3.4 will suffice.
	\begin{answer}
		We have some analysis that we can show.
		Agree that a figure supporting the existence of the dipolar SST anomaly is required -- this was intended to be Fig. 13.
		Discuss with \'{A}ngel whether further analysis is needed here?
		Can put an ENSO time series quite easily in the supplemental figures, it's already downloaded.
	\end{answer}

	\item The derivation of the weather types using the 4 EOFs is not very clear from the text. It will help the typical reader if the method of deriving the 6 weather types was explained in more detail and using physical arguments/analogs.
	\begin{answer}
		OK, we can add some clarity here.
		I'll defer to \'{A}ngel to draw up this argument.
	\end{answer}

	\item It is not clear which of the two types of atmospheric conditions - the synoptic weather type or the large scale type - is more important for the floods? In other words, does a strong El Nino always enhance the occurrence of WT1 and hence always result in a flood? If not are the  WT1 and WT4 more important? I guess the jets are not more important as they occur in no-flood years as well. So then the occurrence of the El Nino in combination with the MJO phases is crucial for the occurrence of these floods? Please explain.
	\begin{answer}
		This is intended to be a central conclusion of the paper, so we should definitely make sure we're being as clear as possible with regard to this point.
		Because of small sample size, it's hard to build a really powerful statistical model relating seasonal features to floods.
		However, we \emph{can} say (i) that ENSO favors jet events; (ii) that the Atlantic dipole favors no-Chaco jet events; (iii) that the MJO modulates the jet; and (iv) that the no-Chaco jet events of WT4 are particularly important for heavy rainfall in this region.
		This is related to Major Comment 1 so let's discuss a bit more.
	\end{answer}
\end{enumerate}

\subsection{Minor Comments}

\begin{enumerate}
	\item I think maps showing the monthly rainfall anomalies in NDJF (or monthly) right in the beginning of the paper are essential. I suggest adding a figure showing these spatial anomaly maps in the main text. These maps will also help validate your choice of analysis regions for weather typing, EOFs etc.
	\begin{answer}
		OK I was a bit hesitant to put in figures before we have described data and methods, but the reviewer thinks it will be really helpful and I agree that it lays the story out more clearly.
		We can alwas cross-reference the methods.
	\end{answer}
	\item Line 83 - westerly wind regime over which region?
	\begin{answer}
		Need to review paper and fill this in
	\end{answer}
	\item Line 127 - Maybe give a reference supporting that the rainfall is mostly nighttime in this region?
	\begin{answer}
		Agreed, will think of an appropriate reference
	\end{answer}
	\item Lines 145 to 151 - some more information about how these seasonal forecasts are produced is needed. For example how much data goes into generating the pdfs? Also what does flexible format mean? is flexible format relevant here? Also what is meant by short record of 2012-2016? do the forecasts themselves span this time period or are the forecasts derived from statistics collected over this time period? I think it is the former, but an explanation in the manuscript will be appreciated and will clear confusion. Since the authors find that the seasonal forecasts performed better than subseasonal forecasts, a greater discussion of how these seasonal forecasts are derived will help.
	\begin{answer}
		I will refer to \'{A}ngel here.
	\end{answer}
	\item Line 181  - Is there a reason for using the 850mb stream function? why not any other level?
	\begin{answer}
		We chose this level because it shows the low-level jet activity and has been shown to be the level at which most moisture transport occurs.
		We could add other levels, but I think we should just state this more clearly in the text.
	\end{answer}
	\item Line 184 - EOFs are in general dependent on the size/location of the domain. Is the EOF analysis performed here sensitive to the domain chosen?
	\begin{answer}
		This is a good point.
		The Leading EOFs are relatively robust to changes of a few degrees in domain.
		These small changes also do not lead to substantial changes in the weather types identified.
		If the domain is made much larger, then other patterns become important.
	\end{answer}
	\item Line 225 - Why are the seasonal forecasts not bias corrected? I see the authors state that the seasonal forecasts are already bias corrected by IRI (Line 268). Maybe bring that line to Line 225? Also the method that IRI uses for this model forecast bias correction hasn't been reported. Do they use some type of MOS?
	\begin{answer}
		Defer to \'{A}ngel here
	\end{answer}
	\item Line 229 - Probability conditional on what?
	\begin{answer}
		I think we can simplify the language around logistic regression.
		Here we mean conditional on a set of predictors.
	\end{answer}
	\item Line 255 - It is not clear how the Kendall's tau rank test helps chose the best result.
	\begin{answer}
		Defer to \'{A}ngel
	\end{answer}
	\item Lines 258 to 263 - Did the authors display this validation in the paper?
	\begin{answer}
		This is how the skill scores are computed.
	\end{answer}
	\item Section on line 309 - Why is this section named climatological drivers? The drivers analyzed here are specific to the 2015-16 flood. Or does this figure show composites over several years irrespective of whether they were flood years or not?
	\begin{answer}
		We should clarify our language here.
		These are for all rainfall events, not just 2015-16.
	\end{answer}
	\item Figure 3 - There are only two rows in this figure (also Figure 7) unlike what the caption says. Maybe the empty space in the figure should be utilized to add grid lines and longitude and latitude values for better readability. Overlay horizontal wind vectors and isobars for improving understandability for the reader.
	\begin{answer}
		As reviewers 1 and 2 pointed out, our captions are incorrect.
		We can overlay wind vectors and isobars -- let's discuss the best way to do that.
	\end{answer}
	\item Lines 317 to 321 - The discussion presented here is hard to follow without wind vectors and isobars in Figure 3. This information can be derived from the stream function but why not present it explicitly? These fields will be especially helpful when the authors comment on the roles of Chaco and No Chaco jets later in the article.
	\begin{answer}
		OK, this is a helpful suggestion.
		Although it makes the figures a bit more cluttered, it's also true that having some more information in the figure will be helpful.
	\end{answer}
	\item Lines 334 to 337 - the relationship between the baroclinic waves and the east-west gradient is not clear. What is the physical meaning of these EOFs? How did the authors deduce the physical relationships between these EOFs and the jets? the explanation given seems insufficient. A better explanation should be provided especially because the subsequent weather types are dependent on these EOFs.
	\begin{answer}
		Reviewer 1 also noted this, see response to Reviewer 1 Comment \#3.
	\end{answer}
	\item Line 371 - weather types 1 and 5 produce rainfall - i don't see WT5 in this period. Also isn't WT5 associated with dry conditions?
	\begin{answer}
		The WT5 days during this period were typically dry days; even with large-scale conditions very favorable to rainfall, we would expect some southerly circulation over the study area.
		Let's clarify our language.
	\end{answer}
	\item Figure 8 - Maybe reiterate in this figure that the forecast was made in November 2015 to remind the reader?
	\begin{answer}
		This would be a helpful thing to add in the figure caption, agreed.
	\end{answer}
	\item Lines 389 to 396 - What constitutes this high odds? That is, what kind of weather conditions captured by the seasonal forecast model could have resulted in these high odds? a seasonal forecast of high LLJ season or a seasonal forecast of a high El Nino and MJO phases 1 to 5?
	\begin{answer}
		We ought to think about what was going on here.
		We may not have a straightforward answer but then at least we ought to say so.
	\end{answer}
	\item Figure 9 - indicate in the figure that the data is not bias corrected.
	\begin{answer}
		Fair point to include.
	\end{answer}
	\item Lines 445 to 453 - How are the PCR and CCA better than others as their high Odds ratio is displaced from the spatial location where the high precipitation was observed?
	\begin{answer}
		``Better'' or ``worse'' is a fairly heuristic description, so let's be clear about what we're saying.
	\end{answer}
	\item Lines 470 to 473 - see comment 14.
	\begin{answer}
		We concur; this point is addressed above, as referenced in comment 14.
	\end{answer}
	\item Lines 474 to 477 - what about the correlations with other EOFs? also why not show a table of all correlations? The correlations reported here are rather small.
	\begin{answer}
		These correlations have been computed so it would be straightforward to present them in the supplemental materials.
		We note that correlations are likely to be lower at daily time steps than monthly ones.
	\end{answer}
	\item Line 476 - what is RMM1
	\begin{answer}
		This is a MJO index, let's make sure we describe this.
	\end{answer}
	\item Paragraph on Line 504 - The purpose of this discussion is not clear. it has too little information about something complex and hence warrants more explanation.
	\begin{answer}
		I agree that we can be a bit more clear about how we describe this.
		Perhaps breaking this section into sub-sections may be helpful?
	\end{answer}
	\item Lines 556 to 557 - The observed …. flooding. The authors have not shown anywhere in the article that - The observed flooding was made more likely by antecedent soil moisture left by the preceding rainy season.
	\begin{answer}
		Reviewer 1 also pointed this out.
		We concur with the reviewer; see our comments to Reviewer 1 comment 4.
	\end{answer}
	\item Panel numbers in figures will be helpful.
	\begin{answer}
		Reviewer 2 also requested these and we will implement them.
	\end{answer}
\end{enumerate}

\subsection{Typos}
\begin{enumerate}
	\item  Line 71 - Change to - correspondence with the exit region of the low-level jets.
	\begin{answer}
		Agreed, correspondence with not correspondence to.
	\end{answer}
	\item Line 361 - change frequency to frequently.
	\begin{answer}
		Agreed
	\end{answer}
	\item Line 418 - Did you mean Figure 10 a?
	\begin{answer}
		Need to add panel numbers in figures and then will make sure we are referencing this correctly.
	\end{answer}
	\item Line 493 - coming from the Pacific
	\begin{answer}
		Yes, that is what we meant.
	\end{answer}
\end{enumerate}
\end{document}
